\documentclass{article}
\def\xcolorversion{2.00}
\def\xkeyvalversion{1.8}

\usepackage[version=0.96]{pgf}
\usepackage{tikz}
\usetikzlibrary{arrows,shapes,snakes,automata,backgrounds,petri}
\usepackage[latin1]{inputenc}
\usepackage{verbatim}

\begin{document}

\begin{comment}
:Title: A Petri-net for Hagen
:Slug: nodetutorial
:Tags: Manual, Petri net, Graphs

This example is from the tutorial: A Petri-net for Hagen.

| Author: Till Tantau
| Source: The PGF/TikZ manual

\end{comment}

\begin{tikzpicture}[node distance=4cm,->,>=stealth',bend angle=45,auto]

  \tikzstyle{place}=[circle,thick,draw=blue!75,fill=blue!20,minimum
  size=6mm,text centered, inner sep=2pt, text width = 2cm]
  \tikzstyle{red place}=[place,draw=red!75,fill=red!20]
  \tikzstyle{transition}=[rectangle,thick,draw=black!75,
  			  fill=black!20,minimum size=4mm]

  \tikzstyle{every label}=[red]

    % Evt 
    \node [place] (rc3)  {RC3};
    \node [place] (rc2) [left of=rc3] {RC2};
    \node [place] (rc1) [left of=rc2] {RC1};
    \node [place] (rc4) [right of=rc3] {RC4};
    \node [place] (rc5) [right of=rc4] {RC5};

   %Community structure 
    \node [place] (rich) [below of=rc3] {Species richness};
    \node [place] (piel) [left of=rich] {Pielou};
    \node [place] (beta)  [left of=piel] {$\beta$-diversity};
    \node [place] (troph) [right of=rich] {Trophic level};
    \node [place] (c)  [right of=troph] {Connectance};

    %bmility
    \node [place] (bm)  [below of=rich] {Total biomass};

    \path 
	(rc3) edge node[anchor=south,above]{$SC_n=0$} (piel)
	(rc3) edge node[anchor=south,above]{$SC_n\neq 0$} (rich)
	(rc3) edge node[anchor=south,above]{$SC_n\neq 0$} (beta)
	(rc3) edge (troph)
	(rc3) edge (c)
	(rc3) edge (bm)

	(rc2) edge (piel)
	(rc2) edge (rich)
	(rc2) edge (beta)
	(rc2) edge (troph)
	(rc2) edge (c)
	(rc2) edge (bm)

	(rc1) edge (piel)
	(rc1) edge (rich)
	(rc1) edge (beta)
	(rc1) edge (troph)
	(rc1) edge (c)
	(rc1) edge (bm)

	(rc4) edge (piel)
	(rc4) edge (rich)
	(rc4) edge (beta)
	(rc4) edge (troph)
	(rc4) edge (c)
	(rc4) edge (bm)

	(rc5) edge (piel)
	(rc5) edge (rich)
	(rc5) edge (beta)
	(rc5) edge (troph)
	(rc5) edge (c)
	(rc5) edge (bm)
	;

    \path 
	(piel)  edge (bm)
	(rich)  edge (bm)
	(beta)  edge (bm)
	(troph) edge (bm)
	(c) edge (bm)
	(bm) edge (bm)
	;


\end{tikzpicture}

\end{document}
