\documentclass{article}
\def\xcolorversion{2.00}
\def\xkeyvalversion{1.8}

\usepackage[version=0.96]{pgf}
\usepackage{tikz}
\usetikzlibrary{arrows,shapes,snakes,automata,backgrounds,petri}
\usepackage[latin1]{inputenc}
\usepackage{verbatim}

\begin{document}

\begin{comment}
:Title: A Petri-net for Hagen
:Slug: nodetutorial
:Tags: Manual, Petri net, Graphs

This example is from the tutorial: A Petri-net for Hagen.

| Author: Till Tantau
| Source: The PGF/TikZ manual

\end{comment}

\begin{tikzpicture}[node distance=3.3cm,>=stealth',bend angle=45,auto]

  \tikzstyle{place}=[circle,thick,draw=blue!75,fill=blue!20,minimum
  size=6mm,text centered, inner sep=2pt, text width = 2cm]
  \tikzstyle{red place}=[place,draw=red!75,fill=red!20]
  \tikzstyle{transition}=[rectangle,thick,draw=black!75,
  			  fill=black!20,minimum size=4mm]

  \tikzstyle{every label}=[red]

  \begin{scope}
    % First net
    \node [place] (evt) {Environment};
    \node [place] (com) [below of=evt] {Community structure}
      edge [pre]                            (evt);
    \node [place] (stab)  [below of=com] {Biomass\\\
	stability}
      edge [pre]                            (com)
      edge [pre, bend left]                            (evt);

  \end{scope}

  \begin{scope}[xshift=6cm]
    % Second net
    \node [place] (evt1) {Environment};
    \node [place] (com1) [below of=evt1] {Community structure}
      edge [pre]                            (evt1);
    \node [place] (bm)  [below of=com1] {Total biomass}
      edge [pre]                            (com1)
      edge [pre, bend right]                (evt1);
  \end{scope}

  \begin{pgfonlayer}{background}
    \filldraw [line width=2cm,join=round,black!10]
      (evt.north  -| stab.east)  rectangle (stab.south  -| stab.west)
      (evt1.north -| bm.east) rectangle (bm.south -| bm.west);
  \end{pgfonlayer}
\end{tikzpicture}

\end{document}
